
\documentclass[aspectratio=169]{beamer}
\usepackage[spanish,es-nodecimaldot]{babel}
\usepackage{amsmath, amssymb, bm}
\usepackage{booktabs}
\usepackage{csquotes}
\usepackage{hyperref}

\usepackage{fontspec}
\setmainfont{EB Garamond}[
  UprightFont = * Regular,
  ItalicFont = * Italic,
  BoldFont = * SemiBold,
  BoldItalicFont = * SemiBold Italic,
]
\setsansfont{EB Garamond}[
  UprightFont = * Regular,
  ItalicFont = * Italic,
  BoldFont = * SemiBold,
  BoldItalicFont = * SemiBold Italic,
]
\usetheme{metropolis}
% No additional code needed here; your fontspec settings are correct.
% If the font is not applying, check that:
% 1. You are compiling with XeLaTeX or LuaLaTeX (not pdfLaTeX).
% 2. The EB Garamond font is installed on your system.
% 3. There are no conflicting packages (e.g., remove any \usepackage{lmodern}, \usepackage[T1]{fontenc}, or similar).
% 4. The theme (if used) does not override fonts. Uncommenting \usetheme{metropolis} may override font settings.


















\title{Juego del Bien Público}
\author{}
\date{\today}

\begin{document}

\begin{frame}
  \titlepage
\end{frame}

\begin{frame}{Contexto}
Muchas personas se benefician de un “bien público” (limpieza de un parque, alumbrado, ciencia abierta). Todos disfrutan el resultado. Nadie puede ser excluido. El incentivo individual es no aportar y esperar que otros paguen. A esto se le llama free-riding.
\end{frame}

\begin{frame}{Definición de bien público}
Un \textbf{bien público} (ejemplo: alumbrado, limpieza de un parque, ciencia abierta) es:
\begin{itemize}
  \item \textbf{No excluible}: todos lo disfrutan.
  \item \textbf{No rival}: el consumo de uno no reduce el de otros.
  \item \textbf{No cooperativo}: cada quien decide cuánto aportar sin coordinación, no se puede obligar a nadie ni negociar
\end{itemize}
El incentivo individual es \emph{no aportar} (free-rider).  
El óptimo social requiere aportaciones positivas.
\end{frame}

\begin{frame}{Reglas del juego}
\begin{itemize}
  \item $e$: dotación que tiene cada jugador.
  \item $n$ jugadores, cada uno recibe dotación $e$.
  \item Cada jugador decide aportar $g_i \in [0,e]$.
  \item Fondo común $G = \sum_j g_j$.
  \item Se multiplica por factor $r$ y se reparte igual.
  \item Pago:
  \[
  \pi_i = e - g_i + \frac{r}{n}G.
  \]
\end{itemize}
Caso típico: $1<r<n \;\Rightarrow\; \text{MPCR (Marginal Per Capita Return)}=\tfrac{r}{n}<1$.
\end{frame}

\begin{frame}{Ejemplo numérico (2 jugadores)}
Parámetros: $n=2$, $e=100$, $r=1.5$, estrategia: aportar $g=50$ o no aportar.  
\begin{center}
\begin{tabular}{c|cc}
 & Aportar & No aportar \\\hline
Aportar & (125,125) & (87.5,137.5) \\
No aportar & (137.5,87.5) & (100,100) \\
\end{tabular}
\end{center}
Interpretación: si ambos aportan ganan más (125), pero siempre hay incentivo a desviarse.
\end{frame}

\begin{frame}{Representación matricial (caso 2 jugadores)}
Estrategias: Aportar (A) una fracción $g_i>0$ o No aportar (N). Sea dotación $e_i$ y retorno $r$ con $\frac{r}{2}\in(0,1)$.
\begin{center}
\scriptsize
\begin{tabular}{c|cc}
 & A & N \\\hline
A & $\big(e_1-g_1+\tfrac{r}{2}(g_1+g_2),\; e_2-g_2+\tfrac{r}{2}(g_1+g_2)\big)$ 
  & $\big(e_1-g_1+\tfrac{r}{2}(g_1+0),\; e_2-0+\tfrac{r}{2}(g_1+0)\big)$\\
N & $\big(e_1-0+\tfrac{r}{2}(0+g_2),\; e_2-g_2+\tfrac{r}{2}(0+g_2)\big)$ 
  & $\big(e_1,\; e_2\big)$
\end{tabular}
\end{center}
\end{frame}



\begin{frame}{Referencias académicas}
\footnotesize
\begin{itemize}
\item Ledyard, J. O. (1995). Public Goods: A Survey of Experimental Research. En J. Kagel \& A. Roth (Eds.), \textit{The Handbook of Experimental Economics}. Princeton University Press.
\item Fehr, E., \& Gächter, S. (2000). Cooperation and Punishment in Public Goods Experiments. \textit{American Economic Review}, 90(4), 980--994.
\item Chaudhuri, A. (2011). Sustaining cooperation in laboratory public goods experiments: A selective survey. \textit{Experimental Economics}, 14, 47--83.
\item Hardin, R. (1982). \textit{Collective Action}. Johns Hopkins University Press.
\end{itemize}
\vspace{0.5em}

\end{frame}

\end{document}
