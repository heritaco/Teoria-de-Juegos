\documentclass[10pt,a4paper]{article} % Compila con lualatex o xelatex

% --- Layout & links ---
\usepackage[margin=1.5in]{geometry}
\usepackage[hidelinks]{hyperref}

% --- Fuentes (texto y matemáticas) ---
\usepackage{unicode-math}
\setmathfont{Latin Modern Math}
\usepackage{fontspec}
\setmainfont{EB Garamond}[
    UprightFont = * Regular,
    ItalicFont = * Italic,
    BoldFont = * SemiBold,
    BoldItalicFont = * SemiBold Italic,
]


% --- Secciones y subsecciones ---
\usepackage{titlesec}

\newfontface\bold{EB Garamond Bold}
\newfontface\bolditalic{EB Garamond  Bold Italic}
\newfontface\extrabold{EB Garamond ExtraBold}

\titleformat{\section}
  {\bold\Large}{\thesection}{0.5em}{}
\titleformat{\subsection}
  {\bold\large}{\thesubsection}{0.5em}{}


% --- Encabezado simple ---
\usepackage{fancyhdr}
\pagestyle{fancy}
\fancyhf{}
\fancyhead[L]{\small\textsc{AI in Financial Services}}
\fancyhead[R]{\small\thepage}
\renewcommand{\headrulewidth}{0.05pt}


% --- Espaciado ---
\usepackage{setspace}
\setstretch{1.25}

% --- Quitar sangría y espacio entre párrafos ---
\usepackage{parskip}
\setlength{\parindent}{0pt}



\begin{document}
\thispagestyle{empty}
\vspace*{-5em}
\begin{center}
  \LARGE\extrabold{Teoría de Juegos}\\
  \vspace{0.5em}
  \scriptsize\textnormal{José Angel, Nuria, Heriberto}
  \vspace{2em}
\end{center}

\begin{center}
  The strategy profile $s^{D}\in S$ is a strict dominant strategy\\ equilibrium if $s_i^{D}\in S_i$ is a strict dominant strategy for all $i\in N$.
\end{center}


\vspace{2em}
\textbf{Prueba por contradicción.} 

Queremos demostrar que si existe un equilibrio en estrategias estrictamente dominantes $s^{D}$, entonces es único. Para ello, supongamos que hay dos perfiles distintos de equilibrio en estrategias estrictamente dominantes, $s^{D}\neq t^{D}$. Elige un jugador $j$ tal que $s^{D}_{j}\neq t^{D}_{j}$.

\vspace{2em}
Como $s^{D}_{j}$ es estrictamente dominante para $j$, para todo $s_{-j}$ y toda $a\neq s^{D}_{j}$, 
$$v_j(s^{D}_{j},s_{-j})>v_j(a,s_{-j}).$$ 
Tomando $s_{-j}=t_{-j}^{D}$ y $a=t^{D}_{j}$, se obtiene
$$v_j(s^{D}_{j},t^{D}_{-j})>v_j(t^{D}_{j},t^{D}_{-j}). \quad (1)$$

\vspace{2em}
Como $t^{D}_{j}$ también sería estrictamente dominante para $j$, para todo $s_{-j}$ y toda $a\neq t^{D}_{j}$,
$$
v_j(t^{D}_{j},s_{-j})>v_j(a,s_{-j}).
$$
Tomando $s_{-j}=s^{D}_{-j}$ y $a=s^{D}_{j}$, se obtiene
$$
v_j(t^{D}_{j},s^{D}_{-j})>v_j(s^{D}_{j},s^{D}_{-j}). \quad (2)
$$


\vspace{2em}
Pero (2) contradice la dominancia estricta de $s^{D}_{j}$ evaluada en $s^{D}_{-j}$, que exige

$$
v_j(s^{D}_{j},s^{D}_{-j})>v_j(t^{D}_{j},s^{D}_{-j}).
$$

\vspace{2em}
La contradicción muestra que no pueden existir dos perfiles distintos. Por tanto, el equilibrio en estrategias estrictamente dominantes es único.


\end{document}
