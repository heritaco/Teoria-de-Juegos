\documentclass[10pt]{article}
\usepackage[a4paper,margin=1.9cm]{geometry}
\usepackage{amsmath,amssymb,mathtools}
\usepackage{enumitem}
\setlist{nosep,leftmargin=*,itemsep=2pt,topsep=2pt}
\newcommand{\br}{\mathrm{BR}}
% ---------- Notation shortcuts ----------
%\newcommand{\Game}{\Gamma}
\newcommand{\Players}{N}
\newcommand{\Si}{S_i}
\newcommand{\Sii}{S_{-i}}
\newcommand{\vi}{v_i}
\newcommand{\BR}{\mathrm{BR}}
\newcommand{\NE}{\text{NE}}
\newcommand{\argmax}{\mathop{\mathrm{arg\,max}}}
\begin{document}

\section*{Competitive Games — Cheat Sheet}

\subsection*{Game types and representation}
\begin{itemize}
  \item \textbf{Strategic settings:} interdependence of actions and payoffs.
  \item \textbf{Noncooperative vs.\ cooperative:} individual actions vs.\ joint actions as shorthand for negotiated contracts.
  \item \textbf{Normal form:} $\Gamma=(N,\{S_i\}_{i=1}^n,\{v_i\}_{i=1}^n)$ with $v_i:S_1\times\cdots\times S_n\to\mathbb R$.
  \item \textbf{Static game, complete information, common knowledge:} simultaneous moves, known actions/outcomes/preferences.
  \item \textbf{Matrix form (2 players, finite):} rows $S_1$, columns $S_2$, entry $(v_1,v_2)$.
\end{itemize}

\subsection*{Zero-sum (matrix) games}
\begin{itemize}
  \item \textbf{Definition:} two-player zero-sum $\Rightarrow$ a single payoff matrix $A=(a_{ij})$ for row player; column gets $-a_{ij}$.
  \item \textbf{Saddle point and value:} if $\max_i\min_j a_{ij}=\min_j\max_i a_{ij}=v$, then $(i^*,j^*)$ is a pure NE and $v$ the value.
  \item \textbf{Constant-sum $\to$ zero-sum:} subtract $k/2$ from both components if $v_1+v_2=k$ (equilibria unchanged).
\end{itemize}

\paragraph{Example: Battle of the Bismarck Sea.}
Model (row=Kenney, col=Imamura, payoff=days of bombing for row):
\[
\begin{array}{c|cc}
 & \text{North} & \text{South}\\\hline
\text{North} & 2 & 2\\
\text{South} & 1 & 3
\end{array}
\]
Row minima: $(2,1)\Rightarrow \max\min=2$ at row North. Column maxima: $(2,3)\Rightarrow \min\max=2$ at column North.
\textbf{Saddle point} $(\text{North},\text{North})$, \textbf{value} $v=2$.

\subsection*{Nonzero-sum canonical games}
\paragraph{Prisoner’s Dilemma.}
\[
\begin{array}{c|cc}
 & C & D\\\hline
C &(-1,-1)&(-10,0)\\
D &(0,-10)&(-9,-9)
\end{array}
\]
$D$ strictly dominates $C$ for both $\Rightarrow$ unique NE $(D,D)$, which is not Pareto-optimal (since $(C,C)$ Pareto-dominates).

\paragraph{Battle of the Sexes.}
\[
\begin{array}{c|cc}
 & A & B\\\hline
A&(2,1)&(0,0)\\
B&(0,0)&(1,2)
\end{array}
\]
No dominance. Best-response intersections $\Rightarrow$ two pure NE: $(A,A)$ and $(B,B)$ (coordination problem).

\paragraph{Cournot duopoly (no costs, inverse demand $p=1-Q$).}
Players choose $q_i\ge 0$. Profits $K_i(q_1,q_2)=q_i(1-q_1-q_2)$ for $q_1+q_2<1$, else $0$.
Best-responses: $q_i=\frac{1-q_j}{2}$. Cournot NE: $q_1=q_2=\tfrac{1}{3}$.
Not Pareto-optimal: $(\tfrac14,\tfrac14)$ raises both firms’ profits.

\subsection*{Nash equilibrium (pure) and how to find it}
\textbf{Definition.} $s^*=(s_1^*,\dots,s_n^*)$ is NE if each $s_i^*\in\arg\max_{t_i\in S_i}v_i(t_i,s_{-i}^*)$.

\textbf{3-step marking for $2\times 2$ (pure NE):}
\begin{enumerate}[label=\arabic*)]
  \item For each column, underline the largest row payoff (row best responses).
  \item For each row, overline the largest column payoff (column best responses).
  \item Any cell with both marks is a pure NE. If strict-dominant profile exists or is the unique IESDS survivor, it is the unique NE.
\end{enumerate}

\subsection*{Pareto optimality}
\textbf{Definition.} $x$ Pareto-dominates $y$ if $v_i(x)\ge v_i(y)\ \forall i$ and $> $ for at least one $i$.
An outcome is Pareto-optimal if it is not Pareto-dominated by any other outcome.
Note: do not confuse Pareto-optimal with “equal payoffs” or symmetry.

\subsection*{Dominance and IESDS}
\textbf{Strict dominance.} $s_i$ strictly dominates $s'_i$ if $v_i(s_i,s_{-i})>v_i(s'_i,s_{-i})$ for all $s_{-i}\in S_{-i}$.
Rational players never play strictly dominated strategies.

\textbf{IESDS algorithm.} Repeatedly delete strictly dominated strategies for any player. The surviving strategy sets $S_i^k$ define the reduced game at round $k$.
If a strict dominant-strategy equilibrium exists, it uniquely survives IESDS.
Profiles that survive IESDS are “iterated-elimination equilibria.”

\subsection*{Extensive form (sequential) pointer}
Sequential Battle of the Sexes: first-mover observed by second; analyze via the game tree (backward reasoning) to select credible outcomes.


\section*{Normal-Form Games — Cheat Sheet}

\subsection*{Core object}
\textbf{Normal-form game}:
\[
\Gamma=\bigl(N,\{S_i\}_{i=1}^n,\{v_i\}_{i=1}^n\bigr),
\quad
v_i:S_1\times\cdots\times S_n\to\mathbb{R}.
\]
Components:
\begin{itemize}
  \item \textbf{Players}: $N=\{1,2,\dots,n\}$.
  \item \textbf{Pure-strategy sets}: $\{S_1,\dots,S_n\}$.
  \item \textbf{Payoff functions}: $\{v_1,\dots,v_n\}$ mapping each joint choice to a real payoff.
\end{itemize}

\subsection*{Simultaneity and outcomes}
\begin{itemize}
  \item \textbf{Simultaneous choice}: each player $i$ selects $s_i\in S_i$ without knowing others’ selections.
  \item \textbf{Strategy profile}: $s=(s_1,\dots,s_n)\in S_1\times\cdots\times S_n$.
  \item \textbf{Realized payoffs}: after choices, player $i$ receives $v_i(s_1,\dots,s_n)$.
\end{itemize}

\subsection*{Strategies}
\begin{itemize}
  \item \textbf{Strategy (informal)}: plan of action aimed at a goal.
  \item \textbf{Pure strategy (formal)}: deterministic plan; the set for player $i$ is $S_i$.
  \item \textbf{Profile of pure strategies}: $s=(s_1,\dots,s_n)$ with $s_i\in S_i$ for all $i$.
\end{itemize}

\subsection*{Canonical example: Prisoner’s Dilemma (normal form)}
Players $N=\{1,2\}$; strategies $S_i=\{C,D\}$.
Representative payoffs:
\[
v_1(C,C)=v_2(C,C)=-1,\quad
v_1(D,D)=v_2(D,D)=-9,
\]
\[
v_1(D,C)=v_2(C,D)=0,\quad
v_1(C,D)=v_2(D,C)=-10.
\]
(Displayed as a $2\times2$ matrix with entries $(v_1,v_2)$.)

\subsection*{Finite games}
A game is \textbf{finite} if the player set is finite and each $S_i$ is finite.

\subsection*{Matrix representation (two-player finite)}
\begin{itemize}
  \item \textbf{Rows}: strategies of player 1; $k$ strategies $\Rightarrow k$ rows.
  \item \textbf{Columns}: strategies of player 2; $m$ strategies $\Rightarrow m$ columns.
  \item \textbf{Cells}: ordered pair $(v_1,v_2)$ at each $(\text{row},\text{column})$.
\end{itemize}

\subsection*{Quick workflow (encode any normal-form problem)}
\begin{enumerate}[label=\arabic*)]
  \item Identify $N$ (decision-makers).
  \item Enumerate $S_i$ (feasible pure actions).
  \item Specify $v_i(\cdot)$ on $S_1\times\cdots\times S_n$.
  \item If $n=2$ and finite, build the payoff matrix with cells $(v_1,v_2)$.
\end{enumerate}



\section*{Dominance \& IESDS — Cheat Sheet}

\subsection*{Notation}
Profile $s=(s_1,\dots,s_n)$ with payoff $v_i(s)=v_i(s_i,s_{-i})$. Opponents’ strategy set $S_{-i}=S_1\times\cdots\times S_{i-1}\times S_{i+1}\times\cdots\times S_n$ and opponents’ profile $s_{-i}\in S_{-i}$.

\subsection*{Pareto baseline}
$x$ Pareto-dominates $y$ if $v_i(x)\ge v_i(y)$ for all $i$ and $>$ for some $i$. Pareto-optimal $\iff$ not dominated.

\subsection*{Strict dominance}
\textbf{Definition.} $s_i'\in S_i$ is strictly dominated by $s_i\in S_i$ if
\[
v_i(s_i,s_{-i})>v_i(s_i',s_{-i})\quad \forall s_{-i}\in S_{-i}.
\]
\textbf{Implication.} A rational player never plays a strictly dominated strategy.

\subsection*{Strictly dominant strategy and equilibrium}
\textbf{Strictly dominant strategy.} $s_i$ is strictly dominant for $i$ if it strictly dominates every other $s_i'\neq s_i$.\\
\textbf{Strict dominant-strategy equilibrium (DS).} $s^D=(s^D_1,\dots,s^D_n)$ where each $s^D_i$ is strictly dominant.\\
\textbf{Uniqueness.} If a game has a strict DS equilibrium $s^D$, then $s^D$ is the unique DS equilibrium.

\subsection*{Weak dominance}
$s_i'$ is weakly dominated by $s_i$ if $v_i(s_i,s_{-i})\ge v_i(s_i',s_{-i})$ for all $s_{-i}$ with strict inequality for some $s_{-i}$. A weakly dominant equilibrium need not be unique.

\subsection*{IESDS: Iterated Elimination of Strictly Dominated Strategies}
\textbf{Rationality premises.} (i) Rational players never play strictly dominated strategies. (ii) Common knowledge of rationality lets players iteratively delete dominated strategies and analyze the reduced game.\\[2pt]
\textbf{Algorithm.} Let $S_i^0=S_i$. For $k=0,1,2,\dots$:
\begin{enumerate}[label=\arabic*)]
\item If some $s_i\in S_i^k$ is strictly dominated, remove all such strategies for all players and set $S_i^{k+1}$ to the survivors.
\item Stop when no further strictly dominated strategies remain. The product $S^K=\prod_i S_i^K$ is the IESDS survivor set.
\end{enumerate}
\textbf{Concept.} Any profile $s^{ES}$ that survives IESDS is an \emph{iterated-elimination equilibrium}.\\
\textbf{Link to DS.} If $s^*$ is a strict DS equilibrium, then $s^*$ uniquely survives IESDS.

\subsection*{Worked elimination pattern (from slides)}
\begin{itemize}
\item Step 1: In a $3\times3$ example, column $C$ is strictly dominated by $R$ for player 2 since $(2>1,\ 6>4,\ 8>6)$ rowwise. Delete $C$.
\item Step 2: In the reduced $3\times2$ game, rows $M$ and $D$ are strictly dominated by $U$ for player 1. Delete $M,D$.
\item Step 3: The residual $1\times2$ game leaves column $L$ strictly dominating $R$ for player 2. Unique survivor profile $(U,L)$ with payoffs $(4,3)$.
\end{itemize}

\subsection*{Workflow on any matrix}
\begin{enumerate}[label=\arabic*)]
\item Scan each player’s strategies for strict dominance; delete.
\item Repeat on the reduced game until no deletions remain (IESDS).
\item If a single profile survives, that is the prediction; if it is also strict DS, it will be the unique Nash equilibrium.
\item Check Pareto efficiency separately; do not confuse \emph{strategies} (solutions) with \emph{payoffs}.
\end{enumerate}



\section*{Nash Equilibrium — Cheat Sheet}

\subsection*{Core definitions}
\textbf{Belief-based:} A Nash equilibrium is a system of beliefs and an action profile where each player plays a best response to their beliefs and beliefs are correct. \\
\textbf{Strategy-based (pure):} A profile $s^*=(s_1^*,\dots,s_n^*)\in S_1\times\cdots\times S_n$ is a Nash equilibrium iff each $s_i^*$ is a best response to $s_{-i}^*$:
\[
v_i(s_i^*,s_{-i}^*) \ge v_i(s'_i,s_{-i}^*) \quad \forall s'_i\in S_i,\ \forall i.
\]

\subsection*{Best responses}
\[
\br_i(s_{-i}) \in \arg\max_{t_i\in S_i} v_i(t_i,s_{-i}).
\]

In continuous-action games, compute first-order conditions (FOC) of $v_i$ w.r.t $t_i$ to obtain $\br_i$, then solve the $n$ equations $\{t_i=\br_i(t_{-i})\}_{i=1}^n$.

\subsection*{Matrix recipe for pure NE (2 players)}
\begin{enumerate}[label=\arabic*)]
\item For each \emph{column} (player 2 fixed), underline the largest payoff of player 1 (row best responses).
\item For each \emph{row} (player 1 fixed), overline the largest payoff of player 2 (column best responses).
\item Any cell with both an under- and an overline is a pure-strategy NE.
\end{enumerate}

\subsection*{Links to other solution concepts}
\textbf{Uniqueness via dominance/IESDS:} If a profile is a strict dominant-strategy equilibrium or the unique survivor of IESDS, then it is the unique NE. \\
\textbf{NE vs.\ Pareto:} NE need not be Pareto-optimal; individual best replies can block social efficiency.

\subsection*{Canonical illustrations (from slides)}
\paragraph{Stag Hunt (Two Kinds of Societies).} Two pure NE: $(S,S)$ and $(H,H)$; $(S,S)$ Pareto-dominates $(H,H)$ (coordination and trust).

\paragraph{Tragedy of the Commons (FOC \& BR method).}
Common resource $K$. Each player $i$ chooses $k_i\ge 0$.
\[
v_i(k_i,k_{-i})=\ln k_i+\ln\!\Big(K-\sum_{j=1}^n k_j\Big).
\]
FOC:
\[
\frac{\partial v_i}{\partial k_i}=\frac{1}{k_i}-\frac{1}{K-\sum_{j=1}^n k_j}=0
\quad\Longrightarrow\quad
k_i=\frac{K-\sum_{j\ne i}k_j}{2}.
\]
For $n=2$:
\[
k_1=\frac{K-k_2}{2},\qquad k_2=\frac{K-k_1}{2}
\ \Longrightarrow\ 
k_1^*=k_2^*=\frac{K}{3}\quad\text{(unique NE)}.
\]
\emph{Pareto benchmark (planner):} maximize $w(k_1,k_2)=\sum_{i=1}^2 v_i$.
FOC yield $k_1^\dagger=k_2^\dagger=\dfrac{K}{4}$.
Conclusion: the NE overuses the commons relative to Pareto optimum.

\subsection*{Quick checklist}
\begin{itemize}
\item Define $N$, $S_i$, $v_i$; write $\br_i$.
\item Discrete case: use 3-step underline/overline test for pure NE.
\item Continuous case: FOC $\Rightarrow$ $\br_i$ equations $\Rightarrow$ solve system.
\item If strict dominance exists for all players, the dominant profile is the unique NE.
\item Do not equate NE with efficiency; check Pareto separately.
\end{itemize}


\newpage
\section*{Game Theory Cheat Sheet (Normal/Matrix Form)}

\subsection*{Model}
\textbf{Normal-form game:} $\Gamma=(\Players,\{S_i\}_{i\in\Players},\{\vi\}_{i\in\Players})$, with $\vi:S_1\times\cdots\times S_n\to\mathbb{R}$. \\
Pure strategies $s=(s_1,\dots,s_n)$, $s_i \in S_i, \forall i$\\
opponents’ set $\Sii=\prod_{j\neq i}S_j$ \\
profile $s_{-i}\in \Sii$.\\
Payoff functions $\vi(s)=\vi(s_i,s_{-i})$. \\
Matrix entries: Each entry in this matrix is an ordered pair $(v_1(s_1 = \{S_i\},s_2),v_2(s)$, where $v_1$ is the payoff to the row player and $v_2$ is the payoff to the column player. \\
\textbf{Static, complete-information, common knowledge:} choices simultaneous; actions, outcomes, and preferences known. % sources

\subsection*{Zero-sum vs.\ nonzero-sum}
\textbf{Zero-sum (2p):} one matrix $A=(a_{ij})$ for row; column payoff $-a_{ij}$. \emph{Saddle point/value:} if $\max_i\min_j a_{ij}=\min_j\max_i a_{ij}=v$, the corresponding cell is a pure NE with game value $v$. Constant-sum reduces to zero-sum by affine shift.
In a zero-sum game, the row player maximizes his minimal payoff $\min_j a_{ij}$ (maximin), while the column player minimizes the maximal drawdown $\max_i a_{ij}$ (minimax). If $\max_i\min_j a_{ij}<\min_j\max_i a_{ij}$, no pure NE exists. % sources: zero-sum basics, saddle point
E. g., Battle of the Bismarck Sea, for the row player (Kenney) vs.\ column (Imamura):
For Kenney the solution is $max(min(2,2),min(1,3))=2$ at North; for Imamura it is $min(max(2,1),max(2,3))=2$ at North. Thus $(\text{North},\text{North})$ is a saddle point with value $2$. % source: Bismarck Sea example

\subsection*{Dominance and IESDS (Iterated Elimination of Strictly Dominated Strategies)}
\textbf{$s'_i$ is strictly dominated by $s_i$} if $\vi(s'_i,s_{-i})<\vi(s_i,s_{-i})$ for all $s_{-i}$. Rational players never use strictly dominated strategies. We say that $s_i \succ s'_i$.\\
\textbf{Strictly dominant strategy:} $s_i \in S_i$ is a strictly dominant strategy for player $i$ if every other strategy is strictly dominated by it $s_i'\neq s_i$.
$v_i(s_i,s_{-i})>v_i(s_i',s_{-i})$ for all $s_i'\neq s_i$, all $s_{-i}$, all $i$. A profile $s^D=(s_1^D,\dots,s_n^D)$ where each $s_i^D$ is strictly dominant is a strict DS equilibrium. If a game has a strict DS equilibrium, the profile is unique if it exists.\\
\textit{Weak dominance:} $s_i$ weakly dominates $s'_i$ if $\vi(s_i,s_{-i})\geq \vi(s'_i,s_{-i})$ for all $s_{-i}$.\\
\textbf{IESDS:} iteratively delete strictly dominated strategies in the reduced game until none remain; the survivors $S_i^K$ define the IESDS survivor set. Any profile $s^{ES}$ that survives IESDS is an iterated-elimination equilibrium. If a strict DS equilibrium $s^*$ exists, it uniquely survives IESDS.

\subsection*{Nash equilibrium}
\textbf{Definition (pure):} $s^*$ is a NE iff $\vi(s_i^*,s_{-i}^*)\ge \vi(t_i,s_{-i}^*)$ for all $t_i\in S_i$, all $i$. Equivalently $s_i^*\in \BR_i(s_{-i}^*)$, where $\BR_i(s_{-i})=\argmax_{t_i\in S_i}\vi(t_i,s_{-i})$.\\
If $s^*$ is either $s^D$ or the \textit{unique} $s^{ES}$ survivor, then $s^*$ is the unique NE. If multiple $s^{ES}$ survive, they are all NE, but there may be other NE as well. If no $s^{ES}$ survives, NE may or may not exist, If a NE exists, it may or may not be Pareto-optimal. If NE exists, a strict DS equilibrium need not exist. If a strict DS equilibrium exists, it is the unique NE. If a NE exists, the IESDS survivor set may be empty, a singleton, or contain multiple profiles.\\
\textbf{Matrix recipe (2p):} underline row best replies column-by-column; overline column best replies row-by-row; double-marked cells are pure NE. A strict DS profile or a unique IESDS survivor is the unique NE.\\
\textbf{Continuous actions:} compute first-order conditions in $t_i$ to get $\BR_i$, then solve the system $t_i=\BR_i(t_{-i})$.

\subsection*{Efficiency}
\textbf{$s$ Pareto-dominates $s$} if $\vi(s)\ge \vi(s')$ $\forall i$ and $>$ for some $i$, in which case \textbf{$s'$ is Pareto-dominated by $s$.} An outcome is \textbf{Pareto-optimal} if it is not Pareto-dominated by any other outcome. 

\subsection*{Canonical micro-examples (minimal forms)}
\paragraph{Bismarck Sea (zero-sum).}
\[
\begin{array}{c|cc}
& \text{North} & \text{South}\\\hline
\text{North} & 2 & 2\\
\text{South} & 1 & 3
\end{array}
\]
Row $\max\min=2$ at North; column $\min\max=2$ at North $\Rightarrow$ saddle point $(\text{North},\text{North})$, value $2$.

\paragraph{Prisoner’s Dilemma.}
\[
\begin{array}{c|cc}
 & C & D\\\hline
C &(-1,-1)&(-10,0)\\
D &(0,-10)&(-9,-9)
\end{array}
\]
$D$ strictly dominates $C$ for both. Unique NE $(D,D)$, not Pareto-optimal.

\paragraph{Battle of the Sexes.}
\[
\begin{array}{c|cc}
 & A & B\\\hline
A&(2,1)&(0,0)\\
B&(0,0)&(1,2)
\end{array}
\]
Two pure NE $(A,A)$ and $(B,B)$ (coordination).

\paragraph{Cournot (costless, $p=1-Q$).} $K_i(q)=q_i(1-\sum_j q_j)$ for $\sum q_j<1$. $\BR_i(q_{-i})=\tfrac{1-q_{-i}}{2}$. For $n=2$: $q_1^*=q_2^*=\tfrac13$ (Cournot NE), not Pareto-optimal.

\paragraph{Tragedy of the Commons (2p).} $v_i(k)=\ln k_i+\ln\!\big(K-\sum_j k_j\big)$.
FOC $\Rightarrow \BR_i(k_{-i})=\dfrac{K-\sum_{j\ne i}k_j}{2}$.
Solve $\Rightarrow k_1^*=k_2^*=\dfrac{K}{3}$ (NE). Planner $\max \sum_i v_i$ yields $k_1^\dagger=k_2^\dagger=\dfrac{K}{4}$ (overuse at NE).

\subsection*{Workflow (one page)}
\begin{enumerate}[leftmargin=*,itemsep=1pt,topsep=2pt]
  \item Specify $(\Players,\{S_i\},\{\vi\})$; build matrix if $n=2$ and finite.
  \item Delete strictly dominated strategies; iterate (IESDS).
  \item Find pure NE by best-reply markings; for continuous actions solve $\{t_i=\BR_i(t_{-i})\}$.
  \item Compare NE to Pareto set.
\end{enumerate}

  \subsection*{Common Knowledge and Complete Information}

  \textbf{Definition (Common Knowledge):} An event $E$ is \emph{common knowledge} if:
  \begin{enumerate}[label=\arabic*)]
    \item Everyone knows $E$,
    \item Everyone knows that everyone knows $E$,
    \item Everyone knows that everyone knows that everyone knows $E$,
    \item and so on ad infinitum.
  \end{enumerate}

  \textbf{Definition (Complete Information):} A game has \emph{complete information} if the following four components are common knowledge among all players:
  \begin{itemize}
    \item All possible actions of all players,
    \item All possible outcomes,
    \item How each combination of actions affects which outcome will materialize,
    \item The preferences of each and every player over outcomes.
  \end{itemize}

   Players are rational: A rational player is one who chooses his action,
 si ∈ Si, to maximize his payoff consistent with his beliefs about
 what is going on in the game.
 Players are intelligent: An intelligent player knows everything about
 the game: the actions, the outcomes, and the preferences of all the
 players.
 Common knowledge: The fact that players are rational and
 intelligent is common knowledge among the players of the game.
 Self-enforcement: It is the core of noncooperative game theory.
 Each player is in control own action, and he will stick to an action
 only if he finds it to be in his best interest

\end{document}
